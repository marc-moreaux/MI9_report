\section{Conclusion}
\label{sec:conclusion}

	We've first tried to gain insight on the dataset to create a model accordingly. After realizing the whole dataset was black-boxed, we created a training process able to learn from two types of feature. The structure and implementation of FTPRL permits us to handle a very large number of one-hot encoded inputs but also, regular input features. Because FTPRL is low regret and converges much faster than gradient descent our process can use FSS to select the features (in a reasonable time frame). 

	The features we created improve the training of our dataset and are re-applicable to other set of real-featured inputs. This generality makes our process re-applicable to other datasets.

	We begun this report stating the desire to produce a method re-applicable to other dataset of unknown nature. I believe that this process answers the initially fixed objective. It is impossible that this process can feat each and every dataset of unknown nature but it can definitely help in the case you want to try a logistic regression on your dataset.