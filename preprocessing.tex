




\section{pre-processing the data}
\label{sec:preprocessing}

	Pre-processing is the fact of transforming an input data into another input data. As it comes before processing the data, it should help this step in doing better work.
	Right now we will see neural network techniques to pre-process the data. Those techniques are described on the "Unsupervised Feature Learning and Deep Learning"\footnote{\url{http://ufldl.stanford.edu/wiki/index.php/UFLDL_Tutorial}} wiki tutorial provided by Stanford.

	\subsection{Data normalization}
		The common methods are: simple rescaling, per-example mean subtraction and feature standardization.\\
		Simple rescaling aims at having all the data dimension on the same scale. You wouldn't want to have an independent feature with values in range [$10^{-5}$,$3*10^{-5}$] and another independent feature in range [$-10^{6}$,$10^{6}$]. Depending on the activation function, the data should be in the range [0,1] or [-1,1].
		Per example mean subtraction consist of removing the mean value of a vector. For instance, if you want to examine the stability of a plane in the air and you have an altitude feature then you may reconsider your ground reference. You can instead consider a mean \textit{plane in the air altitude} as reference to subtract the altitude to.
		Feature standardization consist in two steps: first, setting the input vectors mean to a zero mean and second, setting the variance to a unit variance. To do this, one needs to compute the mean and the variance of the data, then subtract the mean and divide by the variance each data points.



	In our case, we have to preprocess an input matrice so it help our neural network in performing better.
	As mentioned in section \ref{sec:dataset}, the dataset has a large collection of different features (position, cryptography, parsing...). Those features are initially defined as either binaries, integer, or text values. We need to pre-process those to have a fully numerical input vector. For every classes that are not text values, we use feature standardisation by applying the math described previously.





