

\section{pre-processing the data}
\label{sec:preprocessing}

	Pre-processing is the fact of transforming an input data into another input data. As it comes before processing the data, it should help this step in doing better work.
	Right below we present three neural network techniques to pre-process the data. Those techniques are described on the "Unsupervised Feature Learning and Deep Learning"\footnote{\url{http://ufldl.stanford.edu/wiki/index.php/UFLDL_Tutorial}} wiki tutorial provided by Stanford.

	\subsection{Data normalization}
		For data normalization, the common methods are: simple rescaling, per-example mean subtraction and feature standardization.\\
		Simple rescaling aims at having all the data dimension on the same scale. You wouldn't want to have an independent feature with values in range [$10^{-5}$,$3*10^{-5}$] and another independent feature in range [$-10^{6}$,$10^{6}$]. Depending on the activation function, the data should be in the range [0,1] or [-1,1].
		Per example mean subtraction consist of removing the mean value of a vector. For instance, if you want to examine the stability of a plane in the air and you have an altitude feature then you may reconsider your ground reference. You can instead consider a mean \textit{plane in the air altitude} as reference to subtract the altitude to.
		Feature standardization consist in two steps: first, setting the input vectors mean to a zero mean and second, setting the variance to a unit variance. To do this, one needs to compute the mean and the variance of the data, then subtract the mean and divide by the variance each data points.



	In our case, we have to pre-process an input matrix so it help our neural network in performing better.
	As mentioned in section \ref{sec:dataset}, the dataset has a large collection of different features (position, cryptography, parsing...). Those features are initially defined as either binaries, integer, or text values. We need to pre-process those to have a fully numerical input vector.
	Out of the 145 features, we have preprocessed 135 following the feature standardization. The 10 features that haven't been normalized are the hash-text features.



	\subsection{Hash text transformation}
		TODO: explain process to go from hash to integer (dictionary, python code)


	\subsection{analysis of the data}
		TODO: see if some data doesn't help in the learning \\
		$=>$ hash text with stats\\
		$=>$ other with unsupervised learning tech

	\subsection{Network structure}
		TODO: Find one :p \\

		TODO: Special case of hash values (many to few (from unsupervised results))
		As proposed on the competition forum, the text input from the hash code, will be translated into a single binary entry. In other terms, a hash code looking like 'ASddeF54(fd=' will take a binary value '01000'. 



	\subsection{initial values of hidden weights}

		The initial values for the weights of a hidden layer i should be uniformly sampled from a symmetric interval that depends on the activation function. For tanh activation function results obtained in [Xavier10] show that the interval should be 
		$[-\sqrt{\frac{6}{fan_{in}+fan_{out}}},\sqrt{\frac{6}{fan_{in}+fan_{out}}}] $,
		where $fan_{in}$ is the number of units in the $(i-1)-th$ layer, and $fan_{out}$ is the number of units in the $i-th$ layer. For the sigmoid function the interval is $[-4\sqrt{\frac{6}{fan_{in}+fan_{out}}},4\sqrt{\frac{6}{fan_{in}+fan_{out}}}]$.
		This initialization ensures that, early in training, each neuron operates in a regime of its activation function where information can easily be propagated both upward (activations flowing from inputs to outputs) and backward (gradients flowing from outputs to inputs).

		[Xavier10] => Y. Bengio, X. Glorot, Understanding the difficulty of training deep feedforward neuralnetworks, AISTATS 2010

